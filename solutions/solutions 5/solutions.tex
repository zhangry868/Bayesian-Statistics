\documentclass[12pt]{article} 
\input{../../custom}
% Latex Listings language definition/style for Julia
% Source: https://gist.github.com/chi-feng/6589066
%         Github User "chi-feng"
% License: UNKNOWN
% Date accessed: 12/30/2014

% \usepackage{inconsolata} % very nice fixed-width font included with texlive-full
\usepackage{zi4} % EDIT: inconsolata is now accessed through zi4
\usepackage[usenames,dvipsnames]{color} % more flexible names for syntax highlighting colors
\usepackage{listings}
 
\lstset{
basicstyle=\ttfamily, 
columns=fullflexible, % make sure to use fixed-width font, CM typewriter is NOT fixed width
numbers=left, 
numberstyle=\small\ttfamily\color{Gray},
stepnumber=1,              
numbersep=10pt, 
numberfirstline=true, 
numberblanklines=true, 
tabsize=4,
lineskip=-1.5pt,
extendedchars=true,
breaklines=true,        
keywordstyle=\color{Blue}\bfseries,
identifierstyle=, % using emph or index keywords
commentstyle=\sffamily\color{OliveGreen},
stringstyle=\color{Maroon},
showstringspaces=false,
showtabs=false,
upquote=false,
texcl=true % interpet comments as LaTeX
}
 
\lstdefinelanguage{julia}
{
  keywordsprefix=\@,
  morekeywords={
    exit,whos,edit,load,is,isa,isequal,typeof,tuple,ntuple,uid,hash,finalizer,convert,promote,
    subtype,typemin,typemax,realmin,realmax,sizeof,eps,promote_type,method_exists,applicable,
    invoke,dlopen,dlsym,system,error,throw,assert,new,Inf,Nan,pi,im,begin,while,for,in,return,
    break,continue,macro,quote,let,if,elseif,else,try,catch,end,bitstype,ccall,do,using,module,
    import,export,importall,baremodule,immutable,local,global,const,Bool,Int,Int8,Int16,Int32,
    Int64,Uint,Uint8,Uint16,Uint32,Uint64,Float32,Float64,Complex64,Complex128,Any,Nothing,None,
    function,type,typealias,abstract
  },
  sensitive=true,
  morecomment=[l]{\#},
  morestring=[b]',
  morestring=[b]" 
}

\graphicspath{{code/}}
\def\showcommentary{1}


\title{Chapter 5: Solutions to Exercises}
\author{}
\date{}


\begin{document}
\maketitle
\thispagestyle{firststyle}

\subsection*{Exercise 1}

To find the inverse of the c.d.f., set $u = F(x\mid c,\beta)$ and solve for $x$:
\begin{align*}
& u =\exp\Big(-e^{-(x-c)/\beta}\Big)\\
&-\log u = e^{-(x-c)/\beta}\\
&-\log(\log 1/u) = (x-c)/\beta\\
&-\beta\log(\log 1/u)+ c = x.
\end{align*}
Thus, letting
$$G(u) = -\beta\log(\log 1/u)+ c, $$
we have that $G(U)\sim\Gumbel(c,\beta)$ if $U\sim\Uniform(0,1)$.


\subsection*{Exercise 2}

See Figure \ref{figure:cauchy}. See Appendix for source code. Roughly speaking, by inspecting the plots, the approximations seem to be failing to converge---they start approaching 0, but then get ``kicked out'' again.
The issue is that every once in a while, a value of $X_i$ occurs that is so astronomically large compared to all previous values, that it significantly shifts the sample average. 

\begin{figure}
  \begin{center}
    % \includegraphics[trim=0 0.6cm 0 0, clip, width=0.8\textwidth]{cauchy1.png}
    \includegraphics[trim=0 0.6cm 0 0, clip, width=0.8\textwidth]{cauchy2.png}
    \includegraphics[trim=0 0.6cm 0 0, clip, width=0.8\textwidth]{cauchy3.png}
    \includegraphics[trim=0 0.6cm 0 0, clip, width=0.8\textwidth]{cauchy4.png}
    \includegraphics[trim=0 0.6cm 0 0, clip, width=0.8\textwidth]{cauchy5.png}
    % Source: Original work of J. W. Miller.
  \end{center}
  \caption{Lack of convergence of Monte Carlo approximations for Cauchy samples.}
  \label{figure:cauchy}
\end{figure}


\subsection*{Exercise 3}

\subsubsection*{Part (a)}
First, note that, abbreviating $x = x_{1:n}$,
\begin{align*}
\E\Big(\frac{1}{p(x|\btheta)}\Big\vert x\Big) 
& = \int \frac{1}{p(x|\theta)} p(\theta|x)d\theta\\
& = \int \frac{1}{p(x|\theta)} \frac{p(x|\theta) p(\theta)}{p(x)}d\theta\\
& = \frac{1}{p(x)}\int p(\theta) d\theta =\frac{1}{p(x)}
\end{align*}
by Bayes' theorem, and using that $p(x_{1:n}|\theta)>0$ for all $\theta$ to justify the cancellation.

Since this is finite, then letting $\btheta_1,\dotsc,\btheta_N\iid p(\theta|x)$, we have
$$\frac{1}{N}\sum_{i = 1}^N \frac{1}{p(x|\btheta_i)}\longrightarrow \E\Big(\frac{1}{p(x|\btheta)}\Big\vert x\Big)=\frac{1}{p(x)}$$
as $N\to\infty$, with probability 1, by the law of large numbers. Therefore,
$$\Big(\frac{1}{N}\sum_{i = 1}^N \frac{1}{p(x|\btheta_i)}\Big)^{-1}\longrightarrow p(x)$$
as $N\to\infty$, with probability 1.

\subsubsection*{Part (b)}
See Appendix for source code. 
Here are the values of the harmonic mean approximations for 5 independent sets of samples:
\begin{center}
\begin{tabular}{ccccc}
0.109 &
0.097 &
0.106 &
0.100 &
0.094 
\end{tabular}
\end{center}
% [0.10931980367161516,0.0972750607647146,0.10576431583257906,0.10027268715315128,
% 0.0939995577650943]
% 0.038917908100032615
Based on these, we might expect the true value of the marginal likelihood to be somewhere between 0.08 and 0.12.
However, the true value is actually
$$\N(2\mid 0,\lambda^{-1}+\lambda_0^{-1}) = \N(2\mid 0, 1 + 10^2) = 0.0389.$$
The harmonic mean approximations are quite far away from the true value.

\begin{figure}
  \begin{center}
    \includegraphics[width=1\textwidth]{harmonic.png}
    % Source: Original work of J. W. Miller.
  \end{center}
  \caption{Histogram of harmonic mean approximations, compared with the true value (dotted line).}
  \label{figure:harmonic}
\end{figure}

Figure \ref{figure:harmonic} shows a histogram using 1000 independent sets of samples, compared with the true value. The vast majority of the time, the harmonic mean is considerably higher than the true value.


\subsubsection*{Part (c)}
Using $\lambda_0 = 1/100^2$, the harmonic mean gives us
\begin{center}
\begin{tabular}{ccccc}
0.094 &
0.109 &
0.048 &
0.095 &
0.097 
\end{tabular}
\end{center}
while the true value is 0.00399. The harmonic mean is now even more misleading.

% [0.09429627655885037,0.10904854855724887,0.04852115611332322,0.09513155947282682
% ,0.09656096443441156]
% 0.0039884256627033335


\subsection*{Exercise 4}
See Appendix for source code. See module text for plot and discussion.


\subsection*{Exercise 5}

Let $S\subset A$. We have $Z\in S$ if and only if one of the following events occurs:
\begin{align*}
&\{X_1\in S\}\\
&\{X_1\not\in A,X_2\in S\}\\
&\{X_1\not\in A,X_2\not\in A,X_3\in S\}\\
&\vdots
\end{align*}
and these events are mutually exclusive.  Hence, 
\begin{align*}
\Pr(Z\in S) & =\Pr(X_1\in S) +\Pr(X_1\not\in A,X_2\in S) +\Pr(X_1\not\in A,X_2\not\in A,X_3\in S)+\cdots\\
& \overset{\text{(a)}}{=} \sum_{k = 1}^\infty \Big[\prod_{i = 1}^{k-1} \Pr(X_i\not\in A)\Big]\Pr(X_k\in S)\\
& \overset{\text{(b)}}{=} \Pr(X\in S)\sum_{k = 0}^\infty \Pr(X\not\in A)^k\\
& \overset{\text{(c)}}{=} \frac{\Pr(X\in S)}{1-\Pr(X\not\in A)}
 = \frac{\Pr(X\in S)}{\Pr(X\in A)}
 = \Pr(X\in S\mid X\in A)
\end{align*}
where (a) is by independence, (b) since the $X_i$'s are identically distributed, and (c) by the formula for the geometric series: $\sum_{k = 0}^\infty a^k = 1/(1-a)$ for $a\in[0,1)$.



\appendix
\section{Source code in Julia language}

\subsection*{Exercise 2}
\lstinputlisting[language=julia,numbers=none,commentstyle=\tt\color{OliveGreen},firstline=1]{code/cauchy.jl}

\subsection*{Exercise 3}
\lstinputlisting[language=julia,numbers=none,commentstyle=\tt\color{OliveGreen},firstline=1,lastline=32]{code/harmonic.jl}

\subsection*{Exercise 4}
\lstinputlisting[language=julia,numbers=none,commentstyle=\tt\color{OliveGreen},firstline=1]{code/gps-cut.jl}



\end{document}






