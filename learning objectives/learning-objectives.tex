\documentclass[12pt]{article} 
\input{../custom}

\title{Learning Objectives for STA 360/601}
\author{(What you should know for the final.)}
\date{}


\begin{document}
\maketitle

\section{General concepts}
(References: Chapter 1 (Foundations), Hoff chapter 1 and 2.2--2.6, mathematicalmonk PP 2.1--5.5, ML 7.1--7.6)
\begin{itemize}
\setlength\itemsep{0em}
\item Know the difference between probability and statistics.
\item Bayes' theorem: Know the formula for Bayes' theorem, understand what it means, and know how to apply it.
\item Be able to explain the Bayesian approach to statistics.
\item Proportionality: Know the definition, know how to use it to derive conditional distributions such as the posterior, and understand why it works (if two distributions are proportional, they are equal).
\item Cast of characters: Know the definitions (mathematical formulas) of, and understand the purpose of:
\begin{itemize}
\item likelihood / generating distribution
\item prior
\item posterior
\item marginal likelihood
\item posterior predictive
\item loss function
\item posterior expected loss
\item risk / frequent risk
\item integrated risk
\end{itemize}
\end{itemize}


\section{Probability distributions}
(References: Hoff page 253)
Note: You are not expected to memorize the form of each distribution (e.g., the p.d.f./p.m.f.\ and c.d.f.).  A sheet with common distributions will be provided on the exam.
\begin{itemize}
\setlength\itemsep{0em}
\item Understand when a given probability distribution would be appropriate for modeling a given type of data set (e.g., discrete versus continuous, appropriate range of values)
\item Distributions to be familiar with: 
\begin{itemize}
\item Discrete: Geometric, Bernoulli, Binomial, Poisson, Uniform
\item Continuous (univariate): Exponential, Uniform, Gamma, Beta, Pareto, Normal, Inverse Gamma, Cauchy, t-distribution
\item Continuous (multivariate): Multivariate Normal/Gaussian, Wishart, Inverse Wishart
\end{itemize}
\item Be able to give examples of when a distribution would be appropriate/inappropriate.
\end{itemize}

  

\section{Analytical derivations}
(References: Chapters 1, 3, 4, 6, Hoff chapters 3, 5, 6, 7)
\begin{itemize}
\setlength\itemsep{0em}
\item Be able to derive:
\begin{itemize}
\item the posterior,
\item the marginal likelihood, and
\item the posterior predictive,
\end{itemize}
for simple models with conjugate priors, such as:
\begin{itemize}
\item $\Bernoulli(\theta)$ with $\Beta(\theta|a, b)$ prior
\item $\Binomial(n,\theta)$ with $\Beta(\theta|a, b)$ prior
\item $\Geometric(\theta)$ with $\Beta(\theta|a, b)$ prior
\item $\Uniform(0,\theta)$ with $\Pareto(\theta|\alpha,c)$ prior
\item $\Exp(\theta)$ with $\Ga(\theta|a, b)$ prior
\item $\Ga(\alpha,\theta)$ with $\Ga(\theta|a, b)$ prior
\item $\Poisson(\theta)$ with $\Ga(\theta|a, b)$ prior
\item $\N(\mu,\lambda^{-1})$ with $\N(\mu|\mu_0,\lambda_0^{-1})$ prior
\item $\N(\mu,\lambda^{-1})$ with $\Ga(\lambda|a,b)$ prior
\item $\N(\mu,\Lambda^{-1})$ (multivariate) with $\N(\mu|\mu_0,\Lambda_0^{-1})$ prior
\item $\N(\mu,\Lambda^{-1})$ (multivariate) with $\mathrm{Wishart}(\Lambda|S,\nu_0)$ prior
\end{itemize}
\end{itemize}


\section{Decision theory}
(References: Chapter 1 (Foundations), mathematicalmonk ML 3.1--3.4 and 11.1--11.8)
\begin{itemize}
\setlength\itemsep{0em}
\item Understand the decision theoretic setup: state, observation, action, loss.
\item Know the Bayesian approach to decision theory (minimize posterior expected loss).
\item Know the definition of a Bayes procedure.
\item Be able to compute the Bayes procedure (in closed-form when possible) for:
\begin{itemize}
\item 0--1 loss
\item square loss
\item other simple loss functions, such as quadratic functions of the state and action
\end{itemize}
\item Be able to give examples of real-world decision problems that could be addressed using decision theory.
\item Know the definition of admissibility.
\end{itemize}


\section{Exponential families}
(References: Chapter 3 (Exp Fams and Conj Priors), Hoff section 3.3, mathematicalmonk ML 5.1--5.4)
\begin{itemize}
\setlength\itemsep{0em}
\item Know the definition of a one-parameter exponential family.
\item Know the definition of a (multi-parameter) exponential family.
\item Be able to show that a given collection of distributions is an exponential family (one-parameter or multi-parameter).
\item Be able to identify the sufficient statistics function for a given exponential family (one-parameter or multi-parameter).
\item Know the definition of natural form / canonical form, and be able to put a given exponential family into natural form. 
\item Be able to give examples of exponential families.
\end{itemize}


\section{Conjugate priors}
(References: Chapter 3 (Exp Fams and Conj Priors), Hoff section 3.3, mathematicalmonk ML 7.4)
\begin{itemize}
\setlength\itemsep{0em}
\item Know the definition of a conjugate prior family.
\item Be able to show that a given collection of distributions is a conjugate prior for a given generator/likelihood family.
\item Be able to show that mixtures of conjugate priors are conjugate priors.
%\item Be able to show that truncated conjugate priors are conjugate priors.
\item Know how to construct a conjugate prior for an exponential family.
\item Understand that conjugate priors are not unique.
\item Be able to give examples of conjugate priors.
\end{itemize}



\section{Univariate Normal model}
(References: Chapter 4 (Univariate Normal model), Hoff chapter 5, mathematicalmonk ML 7.9--7.10)
\begin{itemize}
\setlength\itemsep{0em}
\item Understand when the Normal model is appropriate.
\item Know the basic properties of the normal distribution: mean, median, mode; symmetric; 95\% probability inside $\pm$1.96 sigma.
\item Know the relationship between the standard deviation, variance, and precision.
\item Know the formula for the distribution of linear combinations of independent normals.
\item Know how to construct a conjugate prior for the mean (Normal--Normal model).
\item Know how to construct a conjugate prior for the mean and precision (NormalGamma--Normal model)
\item Be able to derive the posterior for the Normal--Normal model. (You are not expected to memorize it.)
\item Be able to derive the posterior for the NormalGamma--Normal model. (You are not expected to memorize it.)
\item Be able to choose appropriate values for the prior parameters (hyperparameters).
\item Understand the relationship between the Gamma distribution and Inverse Gamma distribution, and understand how they are used for constructing priors on the precision and variance, respectively.
\item Know that the Normal model is sensitive to outliers.
\end{itemize}


\section{Monte Carlo}
(References: Chapter 5 (Monte Carlo approx), Hoff chapter 4, mathematicalmonk ML 17.1--17.4)
\begin{itemize}
\setlength\itemsep{0em}
\item Know what a (simple) Monte Carlo approximation is.
\item Know what kinds of things can be approximated using Monte Carlo.
\item Understand the advantages and disadvantages of sampling-based methods.
\item Know the standard deviation (i.e., the RMSE) of a Monte Carlo approximation, and know that it represents the rate of convergence.
\item Be able to derive the basic properties of simple Monte Carlo approximations: consistency, unbiasedness, variance, standard deviation, RMSE.
\item Be able to identify the limit of a given Monte Carlo approximation.
\item Know how to construct a Monte Carlo approximation for: posterior probabilities, posterior densities, posterior expected loss, posterior predictive distribution, marginal likelihood.
\item Know the condition under which a simple Monte Carlo approximation is consistent.
\item Be able to give an example for which a simple Monte Carlo approximation is not consistent.
\item Know that the harmonic mean approximation is consistent, but performs poorly.
\end{itemize}



\section{Importance sampling}
(References: Chapter 5 (Monte Carlo approx), mathematicalmonk ML 17.5--17.7)
\begin{itemize}
\setlength\itemsep{0em}
\item Be able to derive the basic importance sampling approximation.
\item Know how to construct a basic importance sampling approximation for a given expectation and a given proposal distribution.
\item Be able to identify the limit of a given importance sampling approximation.
\item Understand the advantages and disadvantages of importance sampling compared to simple Monte Carlo.
\item Be able to derive the basic properties: consistency, unbiasedness, variance, standard deviation, RMSE.
\item Understand the properties that the proposal distribution must have in order to obtain a consistent importance sampling approximation.
\item Understand how to choose the proposal distribution, in order to minimize the approximation error.
\item Know that it is possible to handle unknown normalization constants.
\end{itemize}




\section{Basic techniques for generating samples}
(References: Chapter 5 (Monte Carlo approx), mathematicalmonk ML 17.8--17.14)
\begin{itemize}
\setlength\itemsep{0em}
\item Be able to apply the inverse c.d.f.\ method (derive a formula for transforming $\Uniform(0,1)$ samples into samples from the desired distribution), given the p.d.f.\ of the desired distribution, possibly with an unknown normalization constant.
\item Know that the c.d.f.\ of a continuous distribution transforms samples from that distribution into $\Uniform(0,1)$ samples.
\item Know the precise statement of the inverse c.d.f.\ method, and understand why it is necessary to use the generalized inverse.
\item Understand the rejection principle, and be able to show that it is true.
\item Understand the projection principle, and be able to show that it is true.
\item Understand the rejection sampling procedure, and be able to show that it is true.
\end{itemize}




\section{Gibbs sampling}
(References: Chapter 6 (Gibbs sampling), Hoff chapter 6)
\begin{itemize}
\setlength\itemsep{0em}
\item Know the basic Gibbs sampling algorithm for a distribution $p(\theta_1,\ldots,\theta_k)$.
\item Be able to derive a Gibbs sampler for a given distribution.
\item Know the following terminology: full conditional distribution, sweep/scan, Gibbs updates, burn-in period, good mixing.
\item Understand the advantages and disadvantages of MCMC.
\item Know the definition of a Markov chain.
\item Know how to use the output of an MCMC algorithm to approximating the following quantities:
\begin{itemize}
\setlength\itemsep{0em}
    \item posterior expectations
    \item posterior probability of a given event
    \item posterior densities and c.d.f.s
    \item posterior predictive density and c.d.f.
\end{itemize}
\item Understand why a burn-in period is often needed.
\item Understand the utility of semi-conjugate (a.k.a. conditionally-conjugate) priors in the context of Gibbs sampling.
\item Know how to choose semi-conjugate priors for commonly-used models (e.g., mean and variance for normal distribution).
\item Understand why Gibbs sampling is so useful when using hyperpriors and/or hierarchical models.
\item Understand the concept of data augmentation / auxiliary variables.
\end{itemize}




\section{Priors}
(References: Chapters 3 \& 6, Hoff chapter 9)
\begin{itemize}
\setlength\itemsep{0em}
\item Understand the following terms:
\begin{itemize}
\setlength\itemsep{0em}
    \item conjugate prior
    \item semi-conjugate prior
    \item weakly-informative prior
    \item unit-information prior
    \item g-prior
    \item data-dependent prior
    \item improper prior
    \item non-informative prior
\end{itemize}
\item Understand how the ``posterior'' is defined when using an improper prior (see Chapter 6).
\item Be able to choose prior parameter values that appropriately represent your prior beliefs.
\end{itemize}




\section{MCMC diagnostics}
(References: Chapter 6 (Gibbs sampling), Hoff section 6.6)
\begin{itemize}
\setlength\itemsep{0em}
\item Know how to construct and interpret the following MCMC diagnostics:
\begin{itemize}
\setlength\itemsep{0em}
    \item traceplots
    \item running averages
    \item autocorrelation function (ACF)
    \item scatterplots
    \item estimated posterior densities
\end{itemize}
\item Understand the limitations of MCMC diagnostics (we can only tell when things are going wrong, not when things are going right).
\item Be able to give a specific example in which mixing will appear to be fine according to standard diagnostics, but mixing will actually be very poor.
%\item Be able to give examples in which Gibbs sampling doesn't mix well.
\item Understand why a change of variables can sometimes significantly improve MCMC mixing.
\item Understand why MCMC algorithms sometimes mix poorly on distributions with multiple modes.
\end{itemize}


%\section{Posterior summaries}
%\begin{itemize}
%\setlength\itemsep{0em}
%\item Know the definition of a credible interval (a.k.a. Bayesian confidence interval), and how this differs from a frequentist confidence interval.
%\item 
%\end{itemize}



% \section{Statistical thinking}
% \begin{itemize}
% \setlength\itemsep{0em}
% \item Understand the importance of knowing when some data is missing or censored (see Chapter 6).
% \item 
% \end{itemize}




\section{Multivariate normal/Gaussian distribution}
(References: Hoff chapter 7, mathematicalmonk PP 6.1--6.10)
\begin{itemize}
\setlength\itemsep{0em}
\item Know the density (p.d.f.) of the multivariate normal distribution.
\item Know the definitions of:
\begin{itemize}
\setlength\itemsep{0em}
    \item the covariance and (Pearson's) correlation coefficient between two univariate random variables.
    \item the covariance matrix and the precision matrix.
    \item symmetric positive definite matrix (know at least two definitions).
    \item matrix determinant (know at least two definitions).
    \item matrix inverse.
\end{itemize}
\item Know how to parameterize the covariance matrix of a bivariate normal in terms of the standard deviations $\sigma_1,\sigma_2$ and the correlation coefficient $\rho$.
\item Know the relationship between independence and zero correlation among entries of a multivariate normal vector.
\item Know the affine transformation property.
\item Know how to construct a multivariate normal random vector from i.i.d.\ univariate standard normals.
\item Know how to transform a multivariate random normal vector into i.i.d.\ univariate standard normals (``sphering'').
\item Know how to put a semi-conjugate prior on the mean and covariance matrix.
\item Know how to put a semi-conjugate prior on the mean and precision matrix.
\item Be able to derive the full conditionals for the mean and covariance (or precision) when using a semi-conjugate prior.
\item Recognize the multivariate normal density as $\exp(\text{quadratic})$.
\end{itemize}
% Know that the conditional distributions of a multivariate normal are multivariate normal.
% Know that the marginal distributions of a multivariate normal are multivariate normal.
% Know that the sum of independent multivariate normals is multivariate normal.



\section{Conditional independence relationships and graphical models}
(References: Bishop chapter, mathematicalmonk ML 13.1--13.9)
\begin{itemize}
\setlength\itemsep{0em}
\item Understand that a graphical model does not actually specify a model, but rather a set of conditional independence properties.
\item Understand why graphical models are useful.
\item Know how to write down a directed graphical model (DGM) for a given probabilistic model.
\item Know how to write down the factorization of the joint distribution specified by a DGM.
\item Know what it means for a distribution to respect a given DGM.
\item Understand that one cannot determine \textit{dependence} from a graphical model, only independence.
\item Be able to write down a DGM that is respected by any distribution on 5 variables.
\item Be able to determine the moral graph associated with a given DGM.
\item Know what it means for a distribution to respect a given undirected graphical model (UGM).
\item Know how to determine conditional independence relationships from a UGM.
\end{itemize}




\section{Group comparisons and hierarchical models}
(References: Hoff chapter 8)
\begin{itemize}
\setlength\itemsep{0em}
\item Understand how to define a hierarchical model for data from several groups.
\item Be able to write down an appropriate hierarchical model, given only a verbal description of the data.
\item Be able to give examples in which a hierarchical model would be useful.
\item Understand the concept of ``sharing statistical strength'' (when using a hierarchical model, data from one group helps to infer the parameters of another group).
\item Be able to derive a Gibbs sampler for a hierarchical model with semi-conjugate priors
    (including, but not limited to, the hierarchical normal models described in Hoff 8.3 and 8.5).
\item Understand how deFinetti's theorem can be used to justify modeling the observations with each group, as well as the 
    group parameters, as conditionally i.i.d.
\item Understand the concept of shrinkage, and why it can improve performance.
\end{itemize}




\section{Linear regression}
(References: Hoff chapter 9, mathematicalmonk ML 9.1--10.7)
\begin{itemize}
\setlength\itemsep{0em}
\item Be able to write down the normal linear regression model.
\item Be able to identify when it makes sense to use linear regression in a given problem.
\item Understand how basis functions can be used to model non-linear relationships between the predictor variables $x_i$ and outcomes $y$.
\item Understand what is ``linear'' about linear regression (the mean $\E(Y|x)$ is a linear function of the parameters $\beta$, but not necessarily
    of the predictor variables $x_i$).
\item Be able to derive the ordinary least squares (OLS) estimator of the coefficients (i.e., the maximum likelihood estimator).
\item Know how to define semi-conjugate priors for the coefficients $\beta$ and the variance $\sigma^2$ (or precision $\lambda$).
\item Know how to define a g-prior for the coefficients.
\item Know how to define a unit-information prior for the coefficients.
\item Be able to derive the resulting full conditionals, for each of these types of priors
    (note that the g-prior and unit-information are essentially special cases of the semi-conjugate prior).
\end{itemize}




\section{Bayesian hypothesis testing and model selection/inference}
(References: Lecture notes on course website, Hoff section 9.3, mathematicalmonk ML 12.1--12.4)
\begin{itemize}
\setlength\itemsep{0em}
\item Understand the concept of Bayesian hypothesis testing, and how it differs from frequentist hypothesis testing.
\item Know that Bayesian hypothesis testing, Bayesian model selection, and Bayesian model averaging are all the same thing, mathematically.
\item Be able to give examples of problems in which Bayesian hypothesis testing would make sense.
\item Given a verbal description of a problem, be able to identify when Bayesian hypothesis testing would make sense, and write down an appropriate model.
\item In simple cases, be able to compute the posterior on hypotheses analytically.
\item Understand how decision theory can be used, if one hypothesis must be selected.
\item Know the definition of Bayes factors, and how to interpret them.
\item Know that ``posterior odds = Bayes factor times prior odds''.
\item Be able to compute Bayes factors in simple cases.
\item Understand how Bayes factors (and the posterior on hypotheses) can be strongly affected by the prior on parameters. Understand Lindley's ``paradox''.
\item Understand why improper priors cannot be used when doing Bayesian hypothesis testing. (An improper prior is only defined up to a multiplicative constant.)
\item Know that Bayes factors can be non-monotone in the sample size, and understand why this is the case.
\end{itemize}




\section{Bayesian variable selection}
(References: Hoff section 9.3)
\begin{itemize}
\setlength\itemsep{0em}
\item Understand why it can be advantageous to use a subset of variables, instead of all of them.
\item Understand the basic idea of the backwards elimination procedure (but not necessarily the details).
\item Understand how variable selection can be viewed as a hypothesis testing problem.
\item Be able to write down a model for Bayesian variable selection.
\item Be able to derive a sampler for the posterior, including a Gibbs sampler for the indicator variables (indicating which coefficients are included).
\end{itemize}




\section{Metropolis--Hastings MCMC}
(References: Hoff chapter 10, mathematicalmonk ML 18.1--18.9)
\begin{itemize}
\setlength\itemsep{0em}
\item Be able to write down the Metropolis algorithm and the Metropolis--Hastings (MH) algorithm.
\item Be able to explain some of the advantages/disadvantages of MH versus Gibbs sampling.
\item Be able to show that both the Metropolis algorithm and Gibbs sampling are special cases of MH.
\item Understand the intuition behind the acceptance ratio in the Metropolis algorithm. 
\item Understand the rough idea behind the acceptance ratio in the MH algorithm. 
\item Understand why the choice of proposal distribution affects how well the resulting Markov chain mixes.
\item In simple cases, be able to choose a reasonable proposal distribution.
\item In simple cases, be able to derive the density of the proposal distribution (in order to compute the acceptance ratio).
\end{itemize}




\section{Markov chains and combining MCMC moves}
(References: Hoff chapter 10, mathematicalmonk ML 14.1--14.3, 18.1--18.9)
\begin{itemize}
\setlength\itemsep{0em}
\item Know the mathematical definition of a Markov chain (MC). ``The Markov property''
\item Know the mathematical definition of the following terms, for a time-homogeneous MC taking values in a discrete (countable) space:
    \begin{enumerate}
        \item transition matrix
        \item irreducible MC
        \item aperiodic MC
        \item stationary distribution (a.k.a. invariant distribution)
        \item detailed balance
    \end{enumerate}
\item Be able to give examples of MCs that have each of the following properties, and examples that do not:
    is irreducible, is aperiodic, has a stationary distribution. 
\item In simple cases, be able to identify whether an MC is irreducible and/or aperiodic,
    and be able to identify a stationary distribution (if one exists).
\item Know what the ergodic theorem is, and how it is used in MCMC.
\item Be able to show that having detailed balance implies that the target distribution is a stationary distribution.
\item Understand the intuition behind the definition of stationary distribution, and detailed balance (e.g., the water flow analogy).
\item Be able to show that the MH algorithm induces an MC having detailed balance (for the target distribution).
\item Know how moves (transition matrices) can be combined, using
    \begin{enumerate}
        \item a product of transition matrices, i.e., a repeating cycle of moves (e.g., as in ``fixed-scan'' Gibbs sampling),
        \item a mixture of transition matrices, i.e., a random choice of move (e.g., as in ``random-scan'' Gibbs).
    \end{enumerate}
\item Understand why the choice of move (generally-speaking) cannot depend on the state of the MC.
\item Understand how MH can be used within Gibbs sampling (MH-within-Gibbs).
\item Be able to derive an MH-within-Gibbs sampler for a given model.
\end{itemize}




\section{Advanced Monte Carlo methods}
\begin{itemize}
\setlength\itemsep{0em}
\item In *very* general terms, know the basic idea of the following:
    \begin{enumerate}
        \item quasi-Monte carlo
        \item slice sampling
        \item Hamiltonian MC, a.k.a. hybrid MC
        \item approximate Bayesian computation (ABC)
    \end{enumerate}
\end{itemize}




\section{Mixture models}
(References: mathematicalmonk ML 16.6)
\begin{itemize}
\setlength\itemsep{0em}
\item Know the definition of a mixture.
\item Be able to write down a mixture model for a given problem.
\item Know that mixture models can be used for density estimation, clustering, and latent structure modeling.
\item Know how to derive a Gibbs sampler for a two-component finite mixture model (this was covered in Chapter 6).
\end{itemize}














  


\end{document}






