\documentclass[12pt]{article} 
\input{../../custom}
\graphicspath{{figures/}}
\def\showcommentary{1}


\title{In-class exercise}
\author{}
\date{}


\begin{document}
\maketitle

\subsection*{Instructions}
\begin{itemize}
\item \textbf{Don't look at the solution yet!} This is for your benefit.
\item This exercise must be submitted within 48 hours of the lecture in which it was given. 
\item As long as you do the exercise on time, you get full credit---your performance does not matter.
\item Without looking at the solution, take 5 minutes to try to solve the exercise.
\item Pre-assessment: Write down how correct you think your answer is, from 0 to 100\%.
\item Post-assessment: Now, study the solution and give yourself a ``grade'' from 0 to 100\%.
\item Submit your work on the course website, including the pre- and post- assessments.
\end{itemize}

\subsection*{Exercise}
Show that if $p(\theta)$ and $q(\theta)$ are both p.d.f.s. and
$$p(\theta)\propto q(\theta)$$
then
$$p(\theta) = q(\theta)$$
for all $\theta$.



\newpage
\vfill
\rotatebox{180}{
\begin{minipage}[t][\textheight][t]{\textwidth}
\subsection*{Solution}\scriptsize
The general definition of proportionality is that two functions $f(x)$ and $g(x)$ are proportional if there exist constants $a,b$ which are not both equal to 0 (i.e., either $a$ is nonzero, or $b$ is nonzero, or both of them are nonzero) such that
$$ a f(x) = b g(x)$$
for all $x$. Applying this to the case of $p$ and $q$ above, there are some $a,b$, not both zero, such that
\begin{align}\label{equation:equal}
a p(\theta) = b q(\theta)
\end{align}
for all $\theta$, thus
\begin{align*}
a = a \int p(\theta) d\theta =  \int a p(\theta) d\theta  = \int b q(\theta) d\theta = b \int q(\theta) d\theta = b.
\end{align*}
Therefore, $a=b$, so both $a$ and $b$ are nonzero, and cancelling $a$ with $b$ in Equation \ref{equation:equal} shows that $p(\theta)=q(\theta)$ for all $\theta$.\\
\vspace{2em}~\\
Note: When we are dealing with p.d.f.s or p.m.f.s, both $a$ and $b$ must be nonzero, so the general definition is equivalent to: $f(x)$ and $g(x)$ are proportional if there exists $c$ such that
$$ f(x) = c g(x) $$
for all $x$.
\end{minipage}}

\end{document}






