\documentclass[12pt]{article} 
\input{../../custom}
\graphicspath{{figures/}}
\def\showcommentary{1}


\title{In-class exercise}
\author{}
\date{}


\begin{document}
\maketitle

\subsection*{Instructions}
\begin{itemize}
\item \textbf{Don't look at the solution yet!} This is for your benefit.
\item This exercise must be submitted within 48 hours of the lecture in which it was given. 
\item As long as you do the exercise on time, you get full credit---your performance does not matter.
\item Without looking at the solution, take 5 minutes to try to solve the exercise.
\item Pre-assessment: Write down how correct you think your answer is, from 0 to 100\%.
\item Post-assessment: Now, study the solution and give yourself a ``grade'' from 0 to 100\%.
\item Submit your work on the course website, including the pre- and post- assessments.
\end{itemize}

\subsection*{Exercise}
Suppose $X_1,\dotsc,X_n\iid\N(\theta,\sigma^2)$, and $\theta$ is given a $\N(\mu_0,\sigma_0^2)$ prior.
\begin{itemize}
\item[(a)] When $n=1$, what is the marginal likelihood, $p(x_1)$? %(HINT: There is an easy way and a hard way.)
\item[(b)] When $n>1$, is it true that the marginal likelihood factors as $p(x_{1:n}) = p(x_1)\cdots p(x_n)$?
\end{itemize}


\newpage
\vfill
\rotatebox{180}{
\begin{minipage}[t][\textheight][t]{\textwidth}
\subsection*{Solution}\scriptsize
\subsubsection*{Part (a)}
We could equivalently define the model in the following way: $\btheta\sim\N(\mu_0,\sigma_0^2)$ and $X_i =\btheta + Z_i$, where $Z_i\sim \N(0,\sigma^2)$ independently (that is, independently of each other and of $\btheta$) for $i = 1,\dotsc,n$. Using the rule for linear combinations of independent normals, the marginal distribution of $X_i$ is $\N(\mu_0,\sigma_0^2 + \sigma^2)$.
\subsubsection*{Part (b)}
No. If it were true that $p(x_{1:n}) = p(x_1)\cdots p(x_n)$, this would imply that the $X_i$'s are independent, but we know this is not the case, since, for instance, the posterior predictive $p(x_2|x_1)$ depends on $x_1$.
(Note: The $X_i$'s are \textit{conditionally} independent given $\theta$, but here we are talking about their marginal distribution.)
\end{minipage}}

\end{document}






